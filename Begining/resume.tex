\begin{resume}
	\addchaptertocentry{\resumename} % Add the abstract to the table of contents
%	The Thesis Abstract is written here (and usually kept to just this page). The page is kept centered vertically so can expand into the blank space above the title too\ldots
	\noindent{Un modèle individu-centré (IBM) nommé COSMOSPARK a été développé pour simuler la croissance de la population des Cosmopolites Sordidus, dans les champs de bananiers avec culture associée (maïs). Ce modèle est une extension du modèle COSMOS simulant l'épidémiologie du charançon du bananier \textit{Cosmopolites sordidus}, un ravageur majeur des champs de bananiers. Le modèle COSMOS est basé sur des règles simples de déplacement local des adultes, de ponte des femelles, de développement et de mortalité et d'infestation des larves à l'intérieur des bananiers. COSMOSPARK intègre la prédation des adultes. Le paramètre du modèle a été estimé grâce à l’analyse des données du terrain et le modèle a été validé sur des hypothèses de prédation significative dans des plots avec une diversité complexe. COSMOSPARK a donc été utilisé pour tester différentes dispositions spatiales des bananiers et maïs sur la croissance de \textit{C. sordidus}. La plantation de bananiers intercalée avec  maïs a réduit la population des ravageurs mais aussi le pourcentage de bananiers présentant des attaques sévères. Notre modèle permet d'expliquer le facteur clé de la régulation des populations et de l'épidémiologie de ce ravageur tropical.}
	
	\noindent{Mots clés : IBM, Cosmopolites sordidus, cultures associées, ravageur tropical}
\end{resume}