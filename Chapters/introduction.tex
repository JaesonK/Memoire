% Introduction

\chapter{Introduction Générale} % Main chapter title
\label{introduction} % For referencing the chapter elsewhere, use \ref{Chapter1}
%\addcontentsline{toc}{chapter}{Introduction}
%\minitoclt
\section{Contexte et Justification}
\noindent{La biodiversité, est la variabilité des animaux, des végétaux et des micro-organismes au niveau des gènes, des espèces et des écosystèmes. Elle est nécessaire pour entretenir les fonctions essentielles telles que la structure et les processus de l’écosystème. Le contrôle biologique est à cet effet l’un des plus importants services écosystémiques associés à la biodiversité dans les paysages agricoles (\cite{wilby2002natural} ; \cite{gurr2003multi} ; \cite{fiedler2008special}). Plus de 90 pour cent des insectes potentiellement nuisibles pour les cultures sont régulés grâce aux ennemis naturels provenant de ces exploitations agricoles avec des cultures diversifiées. De nombreuses méthodes de lutte contre les ravageurs, tant traditionnelles que modernes, reposent sur la diversité biologique (\cite{fao2002biodiversity}). Les agroécosystèmes mixtes (cultures intercalaires, systèmes agrosylvopastoraux intégrés), caractérisés surtout par une complexité biologique et structurelle de leur système, sont généralement très productifs (\cite{fao2002biodiversity}).}

\noindent{La gestion naturelle et efficace des pratiques culturales  (association des culture, rotation, jachère, …) modifie les réseaux trophiques des arthropodes et favorise l'abondance des prédateurs généralistes (\cite{pynoo2011predicting}, \cite{jean2011changements}). En effet, on tend plus vers une stabilité des systèmes de cultures associées dans le temps que les monocultures (\cite{mitchell2002effects}), ce qui influe sur la population et la densité des prédateurs généralistes. Dans ces systèmes multi-espèces, la concentration des ressources alimentaires végétales, l’obstacle physique que constitue le mélange de plantes de différentes espèces,  ralentissent la dispersion  des ravageurs ainsi que les possibilités de retrouver ses plantes hôtes (\cite{vinatier2010dynamique};\cite{dassou2016response}.)}

\noindent{Au Bénin la production bananière est concentrée au Sud du pays sur de petites superficies et majoritairement en association avec d’autres cultures dont le maïs et des arbres forestiers et fruitiers (\cite{lay2017dognon}). Malgré les ravages du charançon de bananier \textit{Cosmopolites sordidus} dans la sous-région et aux Antilles françaises, le constat est que ces derniers créent moins de dégâts dans les cultures de bananeraies du Bénin. Ce constat laisse penser un processus de régulation naturel impliquant la biodiversité agricole, forte caractéristiques des champs béninois.}

\noindent{Le charançon du bananier \textit{Cosmopolites sordidus} est le principal ravageur du bananier. Originaire du Sud – Est asiatique (Malaisie et Indonésie), le charançon s’est ensuite diffusé dans toutes les régions tropicales et subtropicales productrices de bananiers et plantains (\cite{gold2001biology}; \cite{mille2006insect}). Les études ont montré que la diversité végétale influence les populations d’insectes en particulier les prédateurs généralistes de C. sordidus  renforçant la régulation biologique (\cite{duyck2011addition} ; \cite{ganry2004diversite} ; \cite{dassou2016response}). Cependant, rare sont celles qui portent sur l’effet de l’organisation paysagère des cultures associées sur la régulation des \cite{C. sordidus} au Bénin.}

\noindent{Ces mécanismes de régulation impliquent, en effet, de nombreux niveaux trophiques incluant généralement les végétaux, les herbivores et les prédateurs d’herbivores. Leurs interactions complexes limitent notre capacité à les prédire et les mobiliser étant donné que différents comportements au niveau individuel peuvent conduire à l'émergence de propriétés au niveau de la population (\cite{grimm2013individual}). Afin de tenir compte des propriétés individuelles des différents niveaux trophiques incluant la biodiversité, nous avons choisi une approche de modélisation basée sur l'individu (IBM) pour aider à expliquer  les modèles de population observés (\cite{winkler2007spread}).}

\section{Problématique}
\noindent{Le charançon \textit{Cosmopolite sordidus} (Germar, 1824) (Coleoptera : Curculionidae) est l’un des principaux ravageurs des bananiers, des bananiers plantains et du genre \textit{Ensete}. La femelle pond des œufs au niveau du bulbe du bananier. Après éclosion, les larves causent d'énormes dégâts en creusant des galeries dans le bulbe. Les galeries creusées par les larves entraînent des perturbations physiologiques et fragilisent la plante qui tombe lors du passage d'un vent (\cite{treverrow1985banana}). Plusieurs méthodes ont été développées pour lutter contre la dispersion de ce dernier. Parmi ces méthodes, nous avons la lutte chimique, courante dans les plantations commerciale, la lutte culturale, efficace pour empêcher l’établissement du charançon mais, ces méthodes ne sont pas toujours optimales et ont des conséquences sur l’environnement et la santé humaine. En plus de ces méthodes de lutte, plusieurs pratiques agricoles sont utilisées pour maintenir et améliorer la reproduction, la survie et l'efficacité des ennemis naturels (prédateurs et agents pathogènes) du charançon. De récentes études sur les agro-écosystèmes ont révélé que les prédateurs généralistes sont les plus impliqués dans le contrôle de charançon du bananier (Dassou et al., 2016) en réduisant ainsi les dégâts du ravageur (\cite{dassou2015facteurs}). Parmi ces prédateurs, les araignées, les forficules et les fourmis constituent les groupes les plus abondants dans ces systèmes de production (\cite{collard2018spatial}, \cite{dassou2016response}). Les systèmes tropicaux diversifiés fournissent des habitats favorables au développement de ces prédateurs. Dans plusieurs agro-écosystèmes à base de bananiers, le maïs est souvent associés pour augmenter le revenu des producteurs. Mais, cette  association constitue également un habitat favorable aux ennemis naturels du charançon du bananier. Dans cette étude, nous avons cherché à comprendre le niveau de prédation de charançons du bananier par les prédateurs généralistes dans différentes configurations d'association de maïs avec le bananier.
}

\section{Questions de recherche}
	\begin{itemize}
	[label=$\bullet$, leftmargin=1cm, parsep=0cm, itemsep=0cm, topsep=0cm ]
	\item Comment la biodiversité des plantes et sa configuration dans les agro-écosystèmes de bananiers influencent la régulation des populations de charançons par les prédateurs?
	\item Comment utiliser des modèles multi-agents pour comprendre la durabilité de la prédation du charançon du bananier dans les agro-écosystèmes de bananiers?
	\end{itemize}

\section{Objectifs de l'étude}
\subsection{Objectif général}
\noindent{Optimiser l’organisation spatiale des parcelles de bananiers et des cultures associées afin de maximiser la régulation de \textit{C. sordidus}.}
\subsection{Objectifs spécifiques}
\begin{itemize}
	[label=$\bullet$, leftmargin=1cm, parsep=0cm, itemsep=0cm, topsep=0cm ]
	\item Monter l’effet de l’organisation spatiale et l’architecture des bananiers associés avec le maïs sur l’abondance du charançon du bananier et sa prédation dans les agro-écosystèmes des bananiers et plantains.
	\item Proposer à partir d'un modèle multi-agent des configurations spatiales d'associations de plantain et de maïs limitant les populations de charançons.
\end{itemize}

\section{Hypothèses de travail}
\begin{enumerate}[{H} 1 :]
	\item La prédation de \textit{C. sordidus} est plus importante dans les cultures associées (on s’intéresse ici à la prédation des adultes susceptibles de se déplacer);
	\item L’efficience de control de \textit{C. sordidus} dépend de son abondance dans le parcellaire hétérogène.
	
\end{enumerate}
%----------------------------------------------------------------------------------------


